\subsection{Emergency Services}

This crew is responsible for handling emergency calls and coordinating communication responses with the appropriate teams. The members of the crew are: an \textbf{emergency call agent} and a \textbf{notification agent}.

\subsubsection{Emergency Call Agent}

The Emergency Call Agent serves as the main point of contact for incoming emergency calls, responsible for collecting essential information from the caller and assessing the severity and nature of the incident.

\begin{itemize}
    \item \textbf{Main task:} To receive, assess, and categorize emergency calls, and subsequently notify the appropriate response units.
    \item \textbf{Tools:} 
    \begin{itemize}
        \item \emph{Data Entry Interface}: A computer system or software for systematically entering and recording emergency details.
        \item \emph{Incident Assessment Tools}: A map to quickly locate and verify the caller's position and assess the scene, potentially including software for prioritizing the severity of the incident if necessary.
    \end{itemize}
    \item \textbf{Type:} Interface agent: This agent emphasizes autonomy and learning to support users. In this case, the Emergency Call Agent acts as a user interface for emergency call handling, gathering caller input and making proactive decisions on incident categorization.
    \item \textbf{Properties:}
    \begin{itemize}
        \item \emph{Reactivity}:  The Emergency Call Agent continuously interacts with the environment (incoming calls) and rapidly assesses each situation to provide an appropriate response.
        \item \emph{Proactiveness}: While primarily reactive, the Emergency Call takes initiative by categorizing and prioritizing incidents, ensuring the most urgent cases receive immediate attention.
        \item \emph{Social Ability}: Capable of basic communication with other agents and potentially human responders, using a structured communication protocol for efficient incident coordination.
        \item \emph{Autonomy}: Operates independently once set up, requiring minimal external input to manage call processing and categorization.
    \end{itemize}
\end{itemize}

\subsubsection{Notification Agent}

A bridge agent that ensures timely and accurate transmission of information between the Emergency Call Agent and emergency response teams.

\begin{itemize}
    \item \textbf{Main task:} To relay details about the incident to the respective emergency teams and manage ongoing updates throughout the response.
    \item \textbf{Tools:}
    \begin{itemize}
        \item \emph{Communication protocols}: Passes the information collected by the Emergency Call Agent to the rest of the crews. It can then continue to relay new updates or details as they come in, ensuring no one is left out of important updates.
    \end{itemize}
    \item \textbf{Type:} Facilitator agent: This type emphasizes communication and interaction, managing the connections and information flow between agents. The Notification Agent serves as the coordinator in the emergency response communication network.
    \item \textbf{Properties:}
    \begin{itemize}
        \item \emph{Reactivity}: Monitors changes from both Emergency Call inputs and emergency response team feedback, adjusting notifications and updates based on dynamic incident progress.
        \item \emph{Social Ability}: Engages in continuous communication with multiple agents in real-time, ensuring all parties are kept informed and aligned on incident status.
        \item \emph{Temporal Continuity}: Remains active throughout the duration of an incident, managing and providing updates until the situation is resolved.
        \item \emph{Flexibility}: Adapts to various communication protocols and priority levels, adjusting its actions according to the severity of the situation and the availability of response teams.
    \end{itemize}
\end{itemize}
