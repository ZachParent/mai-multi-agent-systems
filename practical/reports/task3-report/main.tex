\documentclass[a4paper]{article}
\usepackage[utf8]{inputenc}
\usepackage[margin=0.75in]{geometry}
\usepackage{graphicx}
\usepackage{subcaption}
\usepackage{amsmath}
\usepackage{enumerate} % Import the enumerate package
\usepackage{amsfonts}
\usepackage{hyperref}
\usepackage{float}
\usepackage{listings}
\usepackage{xcolor}
\usepackage{verbatim}


% Listings settings for Python code
\lstset{
    language=Python,
    basicstyle=\ttfamily\small,
    keywordstyle=\color{blue}\bfseries,
    stringstyle=\color{red},
    commentstyle=\color{green!70!black},
    numberstyle=\tiny\color{gray},
    numbers=left,
    stepnumber=1,
    numbersep=8pt,
    showspaces=false,
    showstringspaces=false,
    frame=single,
    breaklines=true,
    breakatwhitespace=true,
    tabsize=4,
    captionpos=b
}

% Define vibrant colors for white background
\definecolor{numb}{rgb}{0, 0.5, 1}       % Bright blue for numbers
\definecolor{punct}{rgb}{0.5, 0.5, 0.5}  % Medium gray for punctuation
\definecolor{delim}{rgb}{0.8, 0.1, 0.1}  % Bright red for delimiters

\lstdefinelanguage{json}{
    basicstyle=\normalfont\ttfamily,
    numbers=left,
    numberstyle=\scriptsize,
    stepnumber=1,
    numbersep=8pt,
    showstringspaces=false,
    breaklines=true,
    frame=lines,
    literate=
	    *{0}{{{\color{numb}0}}}{1}
	    {1}{{{\color{numb}1}}}{1}
	    {2}{{{\color{numb}2}}}{1}
	    {3}{{{\color{numb}3}}}{1}
	    {4}{{{\color{numb}4}}}{1}
	    {5}{{{\color{numb}5}}}{1}
	    {6}{{{\color{numb}6}}}{1}
	    {7}{{{\color{numb}7}}}{1}
	    {8}{{{\color{numb}8}}}{1}
	    {9}{{{\color{numb}9}}}{1}
	    {:}{{{\color{punct}{:}}}}{1}
	    {,}{{{\color{punct}{,}}}}{1}
	    {\{}{{{\color{delim}{\{}}}}{1}
	    {\}}{{{\color{delim}{\}}}}}{1}
	    {[}{{{\color{delim}{[}}}}{1}
	    {]}{{{\color{delim}{]}}}}{1}
	    {true}{{{\color{numb}{true}}}}{1}
	    {false}{{{\color{numb}{false}}}}{1},
}


\title{Emergency Response: A Multi-Agent System}
\author{Sheena Maria Lang, Antonio Lobo Santos, Zachary Parent, \\ María del Carmen Ramírez Trujillo and Bruno Sánchez Gómez}
\date{\today}

\begin{document}

\maketitle
\tableofcontents
\newpage

\section{Introduction}
This report presents the final implementation and results of our multi-agent system (MAS) for emergency response coordination. Building upon our previous designs from Tasks 1 and 2, we have developed a complete, functional system that demonstrates the effectiveness of agent-based approaches in managing complex emergency scenarios.

The system is implemented using CrewAI, a framework that enables the creation and coordination of specialized agent crews. Each crew is designed with specific responsibilities and operates through well-defined processes, ensuring efficient handling of emergency situations. The implementation includes:

\begin{itemize}
    \item \textbf{Emergency Services Crew:} Handles initial emergency assessment and coordination
    \item \textbf{Firefighter Agent Crew:} Manages firefighting resources and operations
    \item \textbf{Medical Services Crew:} Coordinates medical response and hospital resources
    \item \textbf{Public Communication Crew:} Manages public information and communication
\end{itemize}

\paragraph{Report Structure:}
\begin{itemize}
    \item Section \ref{sec:crew_design} details the design and implementation of each crew, including their process definitions and data models
    \item Section \ref{sec:crew_interaction} explains the interaction mechanisms between crews and the overall system flow
    \item Section \ref{sec:testing} presents the results of system testing and validation
    \item Section \ref{sec:conclusion} concludes with insights and potential future improvements
\end{itemize}

The implementation builds upon our previous design while introducing several refinements based on practical considerations and testing results. These modifications are documented and justified throughout the report. The complete source code, along with setup instructions and required input files, is provided in the accompanying project repository.

\section{Crew Design and Implementation}
\label{sec:crew_design}
In this section, we will delve into the key design and implementation decisions for the construction of the Emergency Planner multi-agent system.

\subsection{General Design Principles}
\label{subsec:design_principles}

For each crew in our system, a corresponding Python file is used to instantiate the configuration. These configurations are structured based on CrewAI's YAML schema recommendations for crews\footnote{\url{https://docs.crewai.com/concepts/crews\#yaml-configuration-recommended}}, tasks\footnote{\url{https://docs.crewai.com/concepts/tasks\#yaml-configuration-recommended}}, and agents\footnote{\url{https://docs.crewai.com/concepts/agents\#yaml-configuration-recommended}}.

These files were generated using CrewAI CLI, a universal tool for creating configurations. The following command demonstrates how to initialize a new crew configuration:\newline
\texttt{crewai create crew my\_new\_crew}\footnote{\url{https://docs.crewai.com/concepts/cli\#1-create}}.

The Python file is structured to include the following key elements:

\begin{itemize}
    \item \textbf{Imports}: Required modules, including CrewAI components such as `Agent', `Task', `Crew', and `Process'.
    \item \textbf{Tool Instantiation}: Creation of tools for task-specific functionalities.
    \item \textbf{Agent Configuration}: Agents are defined with specific properties, including their roles, goals, delegation capabilities, verbosity, and parameters for interacting with the language model (e.g., temperature settings).
    \item \textbf{Task Configuration}: Tasks are defined with descriptions, expected outputs, dependencies, and execution modes. These configurations ensure tasks are properly structured and validated. Last task of each crew includes \texttt{output\_pydantic} to ensure that the dictionaries returned to the crew are consistent.
    \item \textbf{Schema Augmentation}: Pydantic schemas can be added to the expected output of tasks using utility functions such as \texttt{add\_schema\_to\_task\_config}. This function modifies the task configuration by appending the schema JSON to the expected output property, ensuring that the LLM can take it into account.
    \item \textbf{Crew Composition}: The crew integrates agents and tasks into a defined process, executed sequentially, to accomplish its objectives.
\end{itemize}


The YAML configurations for agents and tasks specify several general properties:

\paragraph{Agent Configuration:} Agents are defined using YAML properties to specify:
\begin{itemize}
    \item \textbf{Role:} The role the agent plays within the crew.
    \item \textbf{Goal:} The overarching objective or mission assigned to the agent.
    \item \textbf{Delegation Capabilities:} Whether the agent is allowed to delegate tasks.
    \item \textbf{Language Model Parameters:} Specific settings such as the model used and the temperature to control the randomness of the output.
\end{itemize}

\paragraph{Task Configuration:} Tasks are defined in YAML with properties that include:
\begin{itemize}
    \item \textbf{Description:} A clear explanation of the purpose and workflow of the task.
    \item \textbf{Expected Output:} The structure and format of the task output, often validated against a schema.
    \item \textbf{Dependencies:} Other tasks that provide context for the task.
    \item \textbf{Execution Mode:} Specifies whether the task is executed synchronously or asynchronously (e.g., `async\_execution: true`).
\end{itemize}

This modular and schema-driven approach ensures flexibility, reusability, and validation throughout the configuration process.



\subsection{Tools}\label{sec:tools}

In this section, we describe the various tools developed for the Emergency Planner system. These tools are designed to facilitate different aspects of emergency management, including calculating route distances and managing database entries related to hospitals and incidents. Each tool is implemented with specific functionalities to address different requirements in emergency scenarios.

\subsubsection{Route Distance Tool}

\paragraph{Purpose}
The Route Distance Tool calculates the driving route distance between an origin and a destination based on their coordinates. This is essential for determining the quickest routes for emergency response teams.

\paragraph{Implementation}
\begin{itemize}
    \item \textbf{Input:} The tool requires the x and y coordinates of both the origin and destination locations.
    \item \textbf{Execution:} The city map graph is loaded from a GraphML file using the OSMnx \cite{osmnx} package, and the shortest path is calculated using the travel time as the weight. The total distance is then computed.
    \item \textbf{Output:} The tool returns the total driving distance in kilometers.
\end{itemize}

The Route Distance Tool is a critical component to bring the Emergency Planner system closer to the real world. It gives the agents access to accurate geographical information, enabling them to make informed decisions about resource allocation and response times.

\subsubsection{Database Management Tools}

\paragraph{Purpose}
The Database Management Tools include the Hospital Reader, Hospital Updater, and Incident Retrieval tools. These tools manage and update the database entries related to hospitals and incidents, ensuring that the information is current and accurate.

\paragraph{Implementation}
\begin{itemize}
    \item \textbf{Hospital Reader Tool:}
        \begin{itemize}
            \item \textbf{Input:} No input parameters are required for this tool.
            \item \textbf{Execution:} The tool connects to the SQLite \cite{sqlite} database and executes a query to fetch all hospital records.
            \item \textbf{Output:} The tool returns a list of hospitals, including their IDs, locations, and available resources.
        \end{itemize}
    \item \textbf{Hospital Updater Tool:}
        \begin{itemize}
            \item \textbf{Input:} The tool requires the hospital ID, number of beds reserved, number of ambulances dispatched, and number of paramedics deployed.
            \item \textbf{Execution:} The tool connects to the SQLite database and executes an update query to modify the hospital's available resources.
            \item \textbf{Output:} The tool returns a confirmation of the update operation.
        \end{itemize}
    \item \textbf{Incident Retrieval Tool:}
        \begin{itemize}
            \item \textbf{Input:} The tool requires the x and y coordinates of the location, fire severity, fire type, and a summary of the new incident.
            \item \textbf{Execution:} The tool connects to the SQLite database, retrieves related incidents based on proximity, fire severity, and fire type, and inserts the new incident into the database.
            \item \textbf{Output:} The tool returns a list of related incidents.
        \end{itemize}
\end{itemize}

The Database Management Tools are essential for maintaining up-to-date and accurate information on hospitals and incidents. By efficiently managing and updating the database, these tools ensure that the data is consistent throughout crews and runs, thus helping mitigate the hallucinative nature of the LLM-based agents.

To facilitate the utilization of these tools, a script named \texttt{sqlite-init.py}, located in \texttt{src/scripts/}, has been developed to initialize the SQLite database and populate its tables. Additionally, a Makefile is provided to generate a compressed archive of the project's source code, accessible by executing the \texttt{make zip} command.


\subsection{Emergency Services Crew}
\subsection{Emergency Services Crew}

The Emergency Services Crew follows a structured \textbf{sequential process} to ensure prompt handling of emergency calls and
 effective communication with response units. The crew consists of two agents: the \textbf{Emergency Call Agent} and the 
 \textbf{Notification Agent}. Each task is assigned based on the specific role and capabilities of these agents, as indicated in the 
 subsequent description.

\begin{enumerate}
    \item \textbf{Receive and Assess Call.} The \textit{Emergency Call Agent} receives incoming calls and collects relevant details
     about the incident. The information that this agent receives is the answer of the following six questions, and it saves the information
     in a report.
     \begin{itemize}
        \item Is it an indoor or outdoor fire? The answer will be outdoor or indoor.
        \item Where is it? The location is received as coordinates (x,y).
        \item Is anyone inside or trapped? The answer will be an integer number $M$ with the number of trapped people. If $M>0$, then rescues are needed and 
        the \textit{Notification Agent} will detail that to the Fire Fighters Crew.
        \item Are there hazards? The answer will be a boolean: True (yes) or False (no). Examples of hazards could be gas cylinders, 
        chemicals, explosions, etc.
        \item How big is the fire? The fire will be classified as either large (e.g. spreading rapidly), medium (e.g. smoke visible) or 
        small (e.g. small flame), depending on the assess of the \textit{Emergency Call Agent.}
        \item Is anyone injured? How badly? The answer will be a tuple with an integer number $N$ representing the number of injured 
        people and another tuple with the level of risk of each person. They could be classified based on their injuries as 'high risk',
         'risk', or 'out of risk'. If there is no victims then $N=0$ and the second element of the tuple will be an empty list.
    \end{itemize}
    
    \item \textbf{Assign fire priority level.} The \textit{Emergency Call Agent} categorizes the incident and sends initial notifications 
    to the \textit{Notification Agent}. The emergency can be classified in one of the following levels: 
    \begin{itemize}
        \item \textbf{Classifying the Fire Based on the Answers:}  
        The fire is classified into three levels based on the answers:
        \begin{itemize}
            \item \textbf{High Priority:}
            \begin{itemize}
                \item Indoor fire or outdoor fire with significant hazards (e.g., chemicals, gas).
                \item $M > 0$ (people are trapped inside).
                \item Large fire spreading rapidly.
                \item High risk injuries or multiple victims.
            \end{itemize}
            The reason of that is because a fire that is indoors, spreading quickly, with trapped people and significant hazards, 
            is clearly high priority. This combination requires an immediate and large response to prevent further harm or fatalities.
            
        \item \textbf{Medium Priority:}
            \begin{itemize}
                \item Outdoor fire with no hazards or small to medium size fire.
                \item No trapped people, but some people at risk or with moderate injuries.
                \item Medium risk injuries or a small number of victims with manageable injuries.
            \end{itemize}
            It was consider that fires that are moderately dangerous, but not life-threatening, or where injuries are serious but
             not critical, would fall into medium priority. These need quick response, but they don't have the same urgency as a high 
             priority fire.
            
        \item \textbf{Low Priority:}
            \begin{itemize}
                \item Outdoor fire with no hazards and small size.
                \item No trapped people or minimal injuries.
                \item Low risk injuries (minor burns or cuts).
            \end{itemize}
            This fire is less dangerous, with no immediate risk to life or property. A single fire engine may be enough to contain 
            and manage the situation. The priority level is low because the fire does not pose significant harm to anyone involved.
    \end{itemize}

    \item \textbf{Notify other crews.} The \textit{Notification Agent} receives the report and the classification of the fire, and 
    details the information to the appropriate crews, Medical Services Crew and Fire Fighters Crew.
    \begin{itemize}
        \item The following information is provided to the Fire Fighters:
        \begin{itemize}
            \item Location (x,y)
            \item Priority level (low, medium or high).
            \item Number of trapped people $M$.
        \end{itemize}
        \item Regarding the Medical Services, they receive the following information:
        \begin{itemize}
            \item Location (x,y)
            \item Number of injured people $N$.
            \item A list containing the risk of each injured person (out of risk, risk or high risk).
        \end{itemize}
    \end{itemize}
\end{enumerate}

\textbf{Task Dependencies} The Emergency Services Crew operates with interdependent tasks that follow a strict 
\textbf{sequential process}:
\begin{itemize}
    \item \textbf{Assign fire priority level} depends on the completion of \textbf{Receive and Assess Call}.
    \item \textbf{Notify other crews.} depends on the completion of \textbf{Categorize and Notify Response Units}.
\end{itemize}

%\begin{figure}[h]
%    \centering
%    \includegraphics[width=\textwidth]{emergency_services_flow.pdf}
%    \caption{Sequential Process Flow of the Emergency Services Crew with Agent Responsibilities}
%\end{figure}

\subsection{Emergency Services Crew}

The Emergency Services Crew follows a structured \textbf{sequential process} to ensure prompt handling of emergency calls and
 effective communication with response units. The crew consists of two agents: the \textbf{Emergency Call Agent} and the 
 \textbf{Notification Agent}. Each task is assigned based on the specific role and capabilities of these agents, as indicated in the 
 subsequent description.

\begin{enumerate}
    \item \textbf{Receive and Assess Call.} The \textit{Emergency Call Agent} receives incoming calls and collects relevant details
     about the incident. The information that this agent receives is the answer of the following six questions, and it saves the information
     in a report.
     \begin{itemize}
        \item Is it an indoor or outdoor fire? The answer will be outdoor or indoor.
        \item Where is it? The location is received as coordinates (x,y).
        \item Is anyone inside or trapped? The answer will be an integer number $M$ with the number of trapped people. If $M>0$, then rescues are needed and 
        the \textit{Notification Agent} will detail that to the Fire Fighters Crew.
        \item Are there hazards? The answer will be a boolean: True (yes) or False (no). Examples of hazards could be gas cylinders, 
        chemicals, explosions, etc.
        \item How big is the fire? The fire will be classified as either large (e.g. spreading rapidly), medium (e.g. smoke visible) or 
        small (e.g. small flame), depending on the assess of the \textit{Emergency Call Agent.}
        \item Is anyone injured? How badly? The answer will be a tuple with an integer number $N$ representing the number of injured 
        people and another tuple with the level of risk of each person. They could be classified based on their injuries as 'high risk',
         'risk', or 'out of risk'. If there is no victims then $N=0$ and the second element of the tuple will be an empty list.
    \end{itemize}
    
    \item \textbf{Assign fire priority level.} The \textit{Emergency Call Agent} categorizes the incident and sends initial notifications 
    to the \textit{Notification Agent}. The emergency can be classified in one of the following levels: 
    \begin{itemize}
        \item \textbf{Classifying the Fire Based on the Answers:}  
        The fire is classified into three levels based on the answers:
        \begin{itemize}
            \item \textbf{High Priority:}
            \begin{itemize}
                \item Indoor fire or outdoor fire with significant hazards (e.g., chemicals, gas).
                \item $M > 0$ (people are trapped inside).
                \item Large fire spreading rapidly.
                \item High risk injuries or multiple victims.
            \end{itemize}
            The reason of that is because a fire that is indoors, spreading quickly, with trapped people and significant hazards, 
            is clearly high priority. This combination requires an immediate and large response to prevent further harm or fatalities.
            
        \item \textbf{Medium Priority:}
            \begin{itemize}
                \item Outdoor fire with no hazards or small to medium size fire.
                \item No trapped people, but some people at risk or with moderate injuries.
                \item Medium risk injuries or a small number of victims with manageable injuries.
            \end{itemize}
            It was consider that fires that are moderately dangerous, but not life-threatening, or where injuries are serious but
             not critical, would fall into medium priority. These need quick response, but they don't have the same urgency as a high 
             priority fire.
            
        \item \textbf{Low Priority:}
            \begin{itemize}
                \item Outdoor fire with no hazards and small size.
                \item No trapped people or minimal injuries.
                \item Low risk injuries (minor burns or cuts).
            \end{itemize}
            This fire is less dangerous, with no immediate risk to life or property. A single fire engine may be enough to contain 
            and manage the situation. The priority level is low because the fire does not pose significant harm to anyone involved.
    \end{itemize}

    \item \textbf{Notify other crews.} The \textit{Notification Agent} receives the report and the classification of the fire, and 
    details the information to the appropriate crews, Medical Services Crew and Fire Fighters Crew.
    \begin{itemize}
        \item The following information is provided to the Fire Fighters:
        \begin{itemize}
            \item Location (x,y)
            \item Priority level (low, medium or high).
            \item Number of trapped people $M$.
        \end{itemize}
        \item Regarding the Medical Services, they receive the following information:
        \begin{itemize}
            \item Location (x,y)
            \item Number of injured people $N$.
            \item A list containing the risk of each injured person (out of risk, risk or high risk).
        \end{itemize}
    \end{itemize}
\end{enumerate}

\textbf{Task Dependencies} The Emergency Services Crew operates with interdependent tasks that follow a strict 
\textbf{sequential process}:
\begin{itemize}
    \item \textbf{Assign fire priority level} depends on the completion of \textbf{Receive and Assess Call}.
    \item \textbf{Notify other crews.} depends on the completion of \textbf{Categorize and Notify Response Units}.
\end{itemize}

%\begin{figure}[h]
%    \centering
%    \includegraphics[width=\textwidth]{emergency_services_flow.pdf}
%    \caption{Sequential Process Flow of the Emergency Services Crew with Agent Responsibilities}
%\end{figure}


\subsection{Firefighter Agent Crew}
\subsubsection{Implementation}
\subsubsection{Implementation}

\subsection{Medical Services Crew}
\subsubsection{Implementation}
\subsubsection{Implementation}

\subsection{Public Communication Crew}
\subsection{Public Communication Crew}

Structured outputs are crucial for ensuring clarity, consistency, and seamless integration across tasks. Below are the Pydantic models designed for the tasks in the Public Communication Crew process:

\subsubsection{Receive Report Task Output}
\begin{lstlisting}[caption={Pydantic model for Receive Report Task Output}]
class EmergencyReport(BaseModel):
    call_assessment: CallAssessment
    firefighters_response_report: FirefightersResponseReport
    medical_response_report: MedicalResponseReport
    timestamp: datetime
    fire_severity: FireSeverity
    location_x: float
    location_y: float
\end{lstlisting}

\subsubsection{Search Related Cases Task Output}
\begin{lstlisting}[caption={Pydantic model for Search Related Cases Task Output}]
class RelatedCase(BaseModel):
    case_id: int
    fire_severity: FireSeverity
    location_x: float
    location_y: float
    summary: str


class RelatedCases(BaseModel):
    related_cases: List[RelatedCase]
\end{lstlisting}

\subsubsection{Draft Initial Article Task Output}
\begin{lstlisting}[caption={Pydantic model for Draft Initial Article Task Output}]
class DraftArticle(BaseModel):
    title: str
    public_communication_report: str
\end{lstlisting}

\subsubsection{Integrate Additional Information Task Output}
\begin{lstlisting}[caption={Pydantic model for Integrate Additional Information Task Output}]
class IntegratedArticle(BaseModel):
    public_communication_report: str
    integrated_sources: List[str]
\end{lstlisting}

\subsubsection{Review and Authorize Publication Task Output}
\begin{lstlisting}[caption={Pydantic model for Review and Authorize Publication Task Output}]
class ReviewedArticle(BaseModel):
    public_communication_report: str
    mayor_approved: bool
    mayor_comments: str
\end{lstlisting}

\subsubsection{Provide Social Media Feedback Task Output}
\begin{lstlisting}[caption={Pydantic model for Provide Social Media Feedback Task Output}]
class PublicCommunicationReport(BaseModel):
    public_communication_report: str
    mayor_approved: bool
    mayor_comments: str
    social_media_feedback: str
\end{lstlisting}

\paragraph{Summary of Outputs}
\begin{itemize}
    \item \textbf{Receive Report Task Output:} Captures the initial fire incident report relevant details from \textit{Emergency Services Crew}, \textit{Firefighters Crew}, and \textit{Medical Services Crew}.
    \item \textbf{Search Related Cases Task Output:} Retrieves relevant historical cases for contextualization and save this case.
    \item \textbf{Draft Initial Article Task Output:} Records the initial draft content.
    \item \textbf{Integrate Additional Information Task Output:} Updates the draft with integrated sources and revisions.
    \item \textbf{Review and Authorize Publication Task Output:} Specifies the review status and comments from the Mayor.
    \item \textbf{Provide Social Media Feedback Task Output:} Details feedback posted on social media platforms, he can critize the mayor's decission.
\end{itemize}

\subsection{Public Communication Crew}

Structured outputs are crucial for ensuring clarity, consistency, and seamless integration across tasks. Below are the Pydantic models designed for the tasks in the Public Communication Crew process:

\subsubsection{Receive Report Task Output}
\begin{lstlisting}[caption={Pydantic model for Receive Report Task Output}]
class EmergencyReport(BaseModel):
    call_assessment: CallAssessment
    firefighters_response_report: FirefightersResponseReport
    medical_response_report: MedicalResponseReport
    timestamp: datetime
    fire_severity: FireSeverity
    location_x: float
    location_y: float
\end{lstlisting}

\subsubsection{Search Related Cases Task Output}
\begin{lstlisting}[caption={Pydantic model for Search Related Cases Task Output}]
class RelatedCase(BaseModel):
    case_id: int
    fire_severity: FireSeverity
    location_x: float
    location_y: float
    summary: str


class RelatedCases(BaseModel):
    related_cases: List[RelatedCase]
\end{lstlisting}

\subsubsection{Draft Initial Article Task Output}
\begin{lstlisting}[caption={Pydantic model for Draft Initial Article Task Output}]
class DraftArticle(BaseModel):
    title: str
    public_communication_report: str
\end{lstlisting}

\subsubsection{Integrate Additional Information Task Output}
\begin{lstlisting}[caption={Pydantic model for Integrate Additional Information Task Output}]
class IntegratedArticle(BaseModel):
    public_communication_report: str
    integrated_sources: List[str]
\end{lstlisting}

\subsubsection{Review and Authorize Publication Task Output}
\begin{lstlisting}[caption={Pydantic model for Review and Authorize Publication Task Output}]
class ReviewedArticle(BaseModel):
    public_communication_report: str
    mayor_approved: bool
    mayor_comments: str
\end{lstlisting}

\subsubsection{Provide Social Media Feedback Task Output}
\begin{lstlisting}[caption={Pydantic model for Provide Social Media Feedback Task Output}]
class PublicCommunicationReport(BaseModel):
    public_communication_report: str
    mayor_approved: bool
    mayor_comments: str
    social_media_feedback: str
\end{lstlisting}

\paragraph{Summary of Outputs}
\begin{itemize}
    \item \textbf{Receive Report Task Output:} Captures the initial fire incident report relevant details from \textit{Emergency Services Crew}, \textit{Firefighters Crew}, and \textit{Medical Services Crew}.
    \item \textbf{Search Related Cases Task Output:} Retrieves relevant historical cases for contextualization and save this case.
    \item \textbf{Draft Initial Article Task Output:} Records the initial draft content.
    \item \textbf{Integrate Additional Information Task Output:} Updates the draft with integrated sources and revisions.
    \item \textbf{Review and Authorize Publication Task Output:} Specifies the review status and comments from the Mayor.
    \item \textbf{Provide Social Media Feedback Task Output:} Details feedback posted on social media platforms, he can critize the mayor's decission.
\end{itemize}


\section{Crew Interactions and Flow}
\label{sec:crew_interaction}

\subsection{Flow Design and Coordination}
The Emergency Planner Flow is designed to handle emergency situations by coordinating multiple crews. The flow begins with the retrieval of a call transcript, followed by the processing of the call by emergency services. Based on the assessment, firefighters and medical services are dispatched in parallel. Public communication is managed after both teams report or during approval retries. Once the emergency is resolved, the flow concludes with the generation of a comprehensive emergency report, which includes summaries and timestamps from all participating crews.

\subsubsection{State Management}
The system maintains a centralized state using a Pydantic model \footnote{\url{https://docs.crewai.com/concepts/flows\#structured-state-management}}, `EmergencyPlannerState`, which tracks all aspects of the emergency response. This includes the call transcript, assessments, and response reports. The state model ensures type-safe storage and accommodates partial updates.

\subsection{Technical Implementation}
The flow is orchestrated using CrewAI's decorators, which define the sequence and conditions for crew operations. Key flow control points include:

\begin{itemize}
    \item \texttt{@start()}\footnote{\url{https://docs.crewai.com/concepts/flows\#start}} for initiating the call transcript retrieval.
    \item \texttt{@listen()}\footnote{\url{https://docs.crewai.com/concepts/flows\#listen}} for establishing dependencies between operations, such as emergency services processing and the dispatch of firefighters and medical services.
    \item \texttt{@router()}\footnote{\url{https://docs.crewai.com/concepts/flows\#router}} for handling conditional flow control, particularly for public communication approval.
\end{itemize}


\subsubsection{Router Implementation}
The router manages public communication approval, checking if the mayor has approved the communication. If not, it retries up to a maximum count. This ensures that public communication is handled appropriately and efficiently.
This includes the use of \texttt{and\_} and \texttt{or\_} to combine multiple conditions. This is key for the retry mechanism for public communication approval.

\paragraph{Complex Logic for Public Communications}
\texttt{and\_} \footnote{\url{https://docs.crewai.com/concepts/flows\#conditional-logic-and}} and \texttt{or\_} \footnote{\url{https://docs.crewai.com/concepts/flows\#conditional-logic-or}} are used to combine multiple conditions. This is key for the retry mechanism for public communication approval.

\begin{lstlisting}[language=Python]
@listen(or_(and_(firefighters, medical_services), "retry_public_communication"))
def public_communication(self):
    # ...
\end{lstlisting}

\paragraph{Router Logic for Public Communication Approval}
The router emits different messages based on the conditions, either triggering a retry or saving the full emergency report.

\begin{lstlisting}[language=Python]
@router(public_communication)
def check_approval(self):
    logger.info("Checking approval")
    if self.state.public_communication_report.mayor_approved:
        return "save full emergency report"
    elif self.state.mayor_approval_retry_count >= MAX_MAYOR_APPROVAL_RETRY_COUNT:
        return "save full emergency report"
    self.state.mayor_approval_retry_count += 1
    return "retry_public_communication"
\end{lstlisting}

\subsection{Justification of Design Choices}
The design choices are justified by the need for a robust and flexible system that can handle complex emergency scenarios. The use of CrewAI's flow decorators allows for clear and maintainable code, while the parallel processing capabilities ensure timely responses from different crews.


\section{Testing}
\label{sec:testing}

\subsection{Unit Tests}
\label{subsec:unit_tests}

\subsection{Integration Tests}
\label{subsec:integration_tests}

The integration testing of the Emergency Planner system is conducted to ensure that all components, such as the crews and the flow, work together seamlessly to handle emergency scenarios. The testing process involves simulating real-world operations using test transcripts and verifying the generation of comprehensive emergency reports.

\paragraph{Flow Simulation}
The system is initiated using the `crewai flow kickoff` command, which triggers the entire flow from start to finish. This command simulates the intake of an emergency call, processing by various crews, and the generation of a final report. The test transcripts, stored in `call\_transcripts.txt`, provide detailed scenarios that the system must handle, including fires with specific hazards and the presence of injured individuals.

\paragraph{Component Processing and Report Generation}
During the integration tests, the system reads a selected transcript and processes it through the emergency services, firefighters, medical services, and public communication teams. Each crew generates a response report, which is compiled into a full emergency report saved to `emergency\_report.md`. This report includes the call transcript, response summaries, and public communication details, providing a comprehensive overview of the incident and response efforts.

\paragraph{Test Cases}
Two test cases were designed to test the possible scenarios that the system can handle. One with the presence of injured individuals and one without. We also were interested in observing if the retry system worked when the mayor did not immediately approve the public communication message.

\begin{enumerate}
    \item \textbf{Test Case 1: Fire with Injured Individuals}
    \paragraph{Transcript}
    A fire of electrical origin has broken out at coordinates (x: 41.71947, y: 2.84031). The fire is classified as high severity, posing significant danger to the area. Hazards present include gas cylinders and flammable chemicals, further escalating the risk. The fire is indoors, and there are 5 people currently trapped. Additionally, there are 2 injured individuals with minor and severe injuries respectively requiring immediate attention.
    \paragraph{Expected Output}
    The medical team should be dispatched to the scene to attend to the injured individuals. The firefighters should be dispatched to the scene to extinguish the fire. The public communication team should craft a message to inform the public about the emergency.
    \item \textbf{Test Case 2: Large-Scale Fire without Injured Individuals}
    \paragraph{Transcript}
    A gas fire has broken out at coordinates (x: 41.71892, y: 2.84127). The fire is classified as high severity at an abandoned warehouse facility. The hazard present is a storage area containing multiple industrial gas cylinders, which poses a significant risk of explosion. The fire is indoors, but the building has been confirmed empty and unused for several months. No individuals are trapped or at risk, and a security check of the premises has confirmed no squatters or unauthorized persons are present. The fire poses a risk of spreading to neighboring structures if not contained quickly.
    \paragraph{Expected Output}
    No medical assistance is required. The firefighters should be dispatched to the scene to extinguish the fire. The public communication team should craft a message to inform the public about the emergency.
\end{enumerate}

\paragraph{Verification and Readiness}
The integration tests verify that each component of the system functions correctly and that the overall flow produces the expected outputs. By simulating real-world scenarios, the tests ensure that the Emergency Planner is robust, efficient, and ready for deployment in actual emergency situations.
\subsection{Results}
\label{subsec:results}



\section{Conclusion}
\label{sec:conclusion}
% This report presents a comprehensive multi-agent system for emergency response coordination, implementing sophisticated cooperation mechanisms across specialized crews. The key achievements and insights include:

% \begin{itemize}
%     \item \textbf{Process Definition:} Each crew operates with clearly defined sequential workflows:
%     \begin{itemize}
%         \item Emergency Services established structured protocols for initial assessment and crew dispatch
%         \item Firefighters implemented systematic resource allocation and deployment procedures
%         \item Medical Services developed efficient hospital ranking and resource coordination
%         \item Public Communications created an iterative approval workflow with historical case integration
%     \end{itemize}
    
%     \item \textbf{Data Standardization:} Implementation of Pydantic models ensures:
%     \begin{itemize}
%         \item Type-safe data transfer between crews
%         \item Consistent reporting formats
%         \item Structured storage of emergency response states
%     \end{itemize}
    
%     \item \textbf{Coordination Mechanisms:} The system achieves efficient crew interaction through:
%     \begin{itemize}
%         \item Centralized state management
%         \item Parallel processing capabilities
%         \item Sophisticated routing mechanisms
%         \item Retry systems for critical operations
%     \end{itemize}
% \end{itemize}

% The framework demonstrates the effectiveness of structured agent cooperation in emergency response scenarios. Future work could explore the integration of additional specialized crews (such as Forensics) and further optimization of the coordination mechanisms for larger-scale emergencies.

\section{References}
% \bibliography{references}
% \bibliographystyle{plain}
\end{document}
