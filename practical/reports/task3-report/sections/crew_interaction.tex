\section{Crew Interactions and Flow}
\label{sec:crew_interaction}

\subsection{Design and Coordination}
The Emergency Planner Flow is designed to handle emergency situations by coordinating multiple crews. The flow begins with the retrieval of a call transcript, followed by the processing of the call by emergency services. Based on the assessment, firefighters and medical services are dispatched in parallel. Public communication is managed after both teams report or during approval retries. Once the emergency is resolved, the flow concludes with the generation of a comprehensive emergency report, which includes summaries and timestamps from all participating crews.

\subsubsection{State Management}
The system maintains a centralized state using a Pydantic model \footnote{\url{https://docs.crewai.com/concepts/flows\#structured-state-management}}, `EmergencyPlannerState`, which tracks all aspects of the emergency response. This includes the call transcript, assessments, and response reports. The state model ensures type-safe storage and accommodates partial updates.

\subsection{Implementation}
The flow is orchestrated using CrewAI's decorators, which define the sequence and conditions for crew operations. Key flow control points include:

\begin{itemize}
    \item \texttt{@start()}\footnote{\url{https://docs.crewai.com/concepts/flows\#start}} for initiating the call transcript retrieval.
    \item \texttt{@listen()}\footnote{\url{https://docs.crewai.com/concepts/flows\#listen}} for establishing dependencies between operations, such as emergency services processing and the dispatch of firefighters and medical services.
    \item \texttt{@router()}\footnote{\url{https://docs.crewai.com/concepts/flows\#router}} for handling conditional flow control, particularly for public communication approval.
\end{itemize}


\subsubsection{Router}
The router manages public communication approval, checking if the mayor has approved the communication. If not, it retries up to a maximum count. This ensures that public communication is handled appropriately and efficiently.
This includes the use of \texttt{and\_} and \texttt{or\_} to combine multiple conditions. This is key for the retry mechanism for public communication approval.

\paragraph{Complex Logic for Public Communications}
\texttt{and\_} \footnote{\url{https://docs.crewai.com/concepts/flows\#conditional-logic-and}} and \texttt{or\_} \footnote{\url{https://docs.crewai.com/concepts/flows\#conditional-logic-or}} are used to combine multiple conditions. This is key for the retry mechanism for public communication approval.

\begin{lstlisting}[language=Python]
@listen(or_(and_(firefighters, medical_services), "retry_public_communication"))
def public_communication(self):
    # ...
\end{lstlisting}

\paragraph{Router Logic for Public Communication Approval}
The router emits different messages based on the conditions, either triggering a retry or saving the full emergency report.

\begin{lstlisting}[language=Python]
@router(public_communication)
def check_approval(self):
    logger.info("Checking approval")
    if self.state.public_communication_report.mayor_approved:
        return "save full emergency report"
    elif self.state.mayor_approval_retry_count >= MAX_MAYOR_APPROVAL_RETRY_COUNT:
        return "save full emergency report"
    self.state.mayor_approval_retry_count += 1
    return "retry_public_communication"
\end{lstlisting}

\subsection{Justification of Design Choices}
The design choices are justified by the need for a robust and flexible system that can handle complex emergency scenarios. The use of CrewAI's flow decorators allows for clear and maintainable code, while the parallel processing capabilities ensure timely responses from different crews.
