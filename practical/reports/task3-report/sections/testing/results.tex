\subsection{Results}\label{subsec:results}
In this section, we present the results generated by our Emergency Planner system. We will display and analyze the outcomes of each of the two previously introduced test cases. Each test case is designed to evaluate one of the two possible paths of the system's flow, providing insights into its performance and effectiveness under various scenarios.

\subsubsection{Test Case 1: Fire with Injured Individuals}
\begin{lstlisting}[language=]
    # Emergency Report

    ## Call Transcript
    A fire of electrical origin has broken out at coordinates (x: 41.71947, y: 2.84031). The fire is classified as high severity, posing significant danger to the area. Hazards present include gas cylinders and flammable chemicals, further escalating the risk. The fire is indoors, and there are 5 people currently trapped. Additionally, there are 2 injured individuals with minor and severe injuries respectively requiring immediate attention.


    ## Firefighters Response
    *2023-09-01 15:00:00+00:00*
    Firefighters responded to an electrical fire at 41.71947, 2.84031 with high severity. 10 firefighters were deployed and completed activities such as assessing the fire spread, deploying ladder engines, extinguishing the fire, and investigating potential hazards. The trapped people have been identified as 5 individuals.

    ## Medical Response
    *2023-02-20 14:30:00+00:00*
    Medical Response Report for Emergency at Location 41.71947,2.84031. Total Paramedics Deployed: 5, Total Ambulances Dispatched: 1. Estimated Arrival Time: 2023-02-20T15:00:00Z. Equipment Provided: Oxygen Mask for minor injury, Stretcher for severe injury.

    ## Public Communication Report
    # Public Communication Report
    ## Electrical Fire in [Neighborhood]: Safety Update

    ### Incident Details
    As of September 1st, 2023, at 15:00 hours, an electrical fire has been reported at location [41.71947, 2.84031]. The fire's severity is classified as high, with hazardous materials present, including gas cylinders and flammable chemicals.

    ### Response Efforts
    Our firefighting team has deployed 10 personnel to assess the situation, deploy ladder engines, extinguish the fire, and investigate potential hazards. Medical responders have been dispatched with a total of 5 paramedics and 1 ambulance. We urge the public to exercise caution in the surrounding area.

    ### Related Cases
    The following similar incidents were found:
    * Case ID: 22, Severity: High, Location: [41.71947, 2.84031], Summary: The firefighting response was initiated in response to an electrical fire at a location with a high severity rating. The fire posed hazards due to gas cylinders and flammable chemicals, with 5 people trapped inside.
    * Case ID: 23, Severity: High, Location: [41.71947, 2.84031], Summary: The firefighting response was initiated in response to an electrical fire at a location with a high severity rating.
    * Case ID: 24, Severity: High, Location: [41.71947, 2.84031], Summary: Firefighters responded to an electrical fire at 41.71947, 2.84031 with high severity.

    ### Safety Protocols
    In light of this incident, we recommend the following safety protocols:
    * Be aware of your surroundings and exercise caution when in areas where hazardous materials are present.
    * Follow all instructions from local authorities and emergency responders.
    * Stay informed through reliable sources and official updates.

    ### Sources
    This information has been gathered from the following integrated sources:
    - Incident report
    - Related case summaries
    - Official safety protocols

    ### Approved by Mayor
    True

    ### Mayor's Comments
    The article appears to be well-researched and accurately reflects the incident details. However, I would like to see more specificity in the safety protocols section, perhaps highlighting concrete actions residents can take to ensure their own safety. Additionally, it might be beneficial to include a statement about support services available for those affected by the fire.

    ### Social Media Feedback
    BREAKING: Electrical fire in [Neighborhood]! Our firefighting heroes are on the scene, but let's get one thing straight... Why are there THREE identical incidents reported from the same location? Were they separate fires or just a case of copy-paste chaos? Can we get some more transparency on this? Also, great job on having 10 firefighters and 5 paramedics ready to roll in seconds. Your dedication is fire (pun intended)! Just remember, safety protocols are like pizza - even when they're bad, they can still bring people together! #FireSafety #TransparencyMatters

\end{lstlisting}

Overall, the system successfully processed the emergency report, dispatched the appropriate crews, and generated the necessary public communication report. The system's ability to handle complex scenarios involving multiple crews and diverse responsibilities was demonstrated effectively. The generated reports have the expected structure, and successfully provided detailed information about the incident, response efforts, and safety protocols.

However, there are some visible flaws in the generated reports that indicate the system still has room for improvement. We highlight mainly two issues that need to be addressed:
\begin{itemize}
    \item The timestamps in the reports are inconsistent. This is due to the agents' inability to access any centralized clock or time service. As a result, each agent generates its own timestamps, leading to discrepancies in the final reports. This could be an important feature in a real-wolrd scenario where the agent crews have to coordinate their actions based on a common timeline, and it could be easily implemented with a custom tool, were the agents to be deployed in a real enviroment.
    \item The Public Communication Report contains placeholder text, such as \texttt{[Neighborhood]}. This is due to the lack of contextual geographical information in the emergency report. Only the coordinates are provided, which can easily be used to calculate distances but do not provide any meaningful information about the region where the incident is taking place. This could be addressed by including more detailed information in the report or, more realistically, by integrating the system with a geolocation service that can provide additional context based on the coordinates. To implement the latter would not be a trivial task.
\end{itemize}

\subsubsection{Test Case 2: Large-Scale Fire without Injured Individuals}

\begin{lstlisting}[language=]
    # Emergency Report

    ## Call Transcript
    A gas fire has broken out at coordinates (x: 41.71892, y: 2.84127). The fire is classified as high severity at an abandoned warehouse facility. The hazard present is a storage area containing multiple industrial gas cylinders, which poses a significant risk of explosion. The fire is indoors, but the building has been confirmed empty and unused for several months. No individuals are trapped or at risk, and a security check of the premises has confirmed no squatters or unauthorized persons are present. The fire poses a risk of spreading to neighboring structures if not contained quickly.


    ## Firefighters Response
    *2023-09-18 15:00:00+00:00*
    A high-severity gas fire was reported at a location with trapped people. We deployed 12 firefighting combatants to the scene and initiated suppression and ventilation activities. The estimated arrival time was 2023-09-18T14:30:00Z.

    ## Medical Response
    *2023-09-18 15:00:00+00:00*
    Medical services not required

    ## Public Communication Report
    **Public Communication Report**

    ### High-Severity Gas Fire Reported in Urban Area, Immediate Safety Information Provided

    A high-severity gas fire has been reported at a location with trapped people. The fire department has responded with 12 firefighting combatants and initiated suppression and ventilation activities. There are no injuries or medical services required. Citizens are advised to avoid the area until further notice.

    ### Similar Past Incidents

    *   **Case ID 32:** A high-severity gas fire was reported at a location with trapped people on September 25, 2023, at 14:30:00Z. Eight combatants were deployed to extinguish the fire using foam and secure nearby gas cylinders.
    *   **Case ID 31:** Firefighters responded to a high-severity gas fire at 41.71892, 2.84127 on September 25, 2023, at 14:30:00Z. Eight combatants were deployed to extinguish the fire with foam and secure nearby gas cylinders within estimated arrival time.
    *   **Case ID 30:** Firefighters responded to a high-severity gas fire at 41.71892, 2.84127 on September 25, 2023.

    ### Recommended Safety Protocols

    To ensure public safety during high-severity incidents like this one:
    1.  **Avoid the Area**: Citizens are advised to avoid the area where the incident is taking place until further notice.
    2.  **Follow Evacuation Orders**: If evacuation orders are issued, please follow them promptly and in an orderly manner.
    3.  **Stay Informed**: Keep tuned to local news and official announcements for updates on the situation.

    **Integrated Sources**

    *   [Source 1](https://example.com/source1): Official report by the fire department
    *   [Source 2](https://example.com/source2): News article about a similar incident

    Note: This response has been crafted to provide clarity, coherence, and credibility, while keeping the community well-informed about unfolding events and safety measures.

    ### Approved by Mayor
    True

    ### Mayor's Comments
    The article appears to be well-researched and provides timely information to the public. However, I recommend adding more details about the cause of the gas fire and any measures being taken by local authorities to prevent such incidents in the future. Additionally, it would be beneficial to include a section on how citizens can prepare themselves for similar emergencies. Overall, this article aligns with city policies and serves the community's best interests.

    ### Social Media Feedback
    Just when we thought gas fires were a thing of the past, Mother Nature decides to remind us she's still in charge! Kudos to our brave firefighters for containing this high-severity incident without any injuries. A shoutout to the swift communication from the fire department, keeping everyone informed and safe!

    Constructive feedback: While the response was top-notch, let's hope for a few less similar past incidents next time ( Case ID 32-30, we're looking at you). Seriously though, it's great to see integrated sources being utilized. Keep up the fantastic work, emergency responders! #gasfire #emergencyresponse #safetyfirst
\end{lstlisting}

In this case, the system appropriately identified that medical services were not required and generated the rest of the corresponding reports. As was to be expected, the reports contain the necessary information about the incident, response efforts, and safety protocols. The system's ability to handle scenarios where the medical services crew is not needed was demonstrated effectively.

Similarly to the previous test case, we observe some issues in the behaviour of the agents that need to be addressed:
\begin{itemize}
    \item The estimated arrival time of the firefighters is not consistent with the timestamp of the report. This is likely caused by the fact that the estimated times of arrival are not calculated programatically and are instead generated by the agents themselves, based on the timestamps and the distance to the emergency. This could be addressed by including the arrival time calculation into the Route Distance Tool, so that it is calculated automatically and consistently.
    \item The Public Communication Report again contains placeholder text. In this case it is the \texttt{Integrated Sources} section, which is not filled with any true sources. A fix for this issue would involve formalizing the sources of information that the system uses to generate the reports, and ensuring that they are included in the final reports. This could be achieved by integrating the system with external data sources that provide relevant information about the incidents.
\end{itemize}