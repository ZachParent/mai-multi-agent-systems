\subsubsection{Pydantic Outputs}
Structured outputs ensure accurate information handling and effective communication within the Emergency Services Crew. Below 
are the Pydantic models designed for each task's output.

\paragraph{Receive and Assess Call Task Output}
\begin{lstlisting}[caption={Pydantic model for Receive and Assess Call Task Output}] 
class EmergencyDetails(BaseModel):
    fire_type: FireType  # Type of fire (e.g., ordinary, electrical, gas, etc.)
    location: Location  # Coordinates (x, y)
    injured_details: List[InjuryType]  # List of risk levels of injured people
    fire_severity: FireSeverity  # Severity of fire: low, medium, or high
    hazards: List[HazardType]  # Hazards present, e.g., gas cylinders, chemicals
    indoor: bool  # True if fire is indoor, False otherwise
    trapped_people: int  # Number of people trapped (0 if none)
\end{lstlisting}

\paragraph{Notify Other Crews Task Output}
\begin{lstlisting}[caption={Pydantic model for Notify Other Crews Task Output}] 
class CallAssessment(BaseModel):
    fire_type: FireType
    location: Location
    injured_details: List[InjuryType]
    fire_severity: FireSeverity
    hazards: List[HazardType]
    indoor: bool
    trapped_people: int
    medical_services_required: bool  # True if medical services are required, False otherwise
\end{lstlisting}

\paragraph{Summary of Outputs}
The Pydantic outputs for the \textit{Emergency Services Crew} ensure structured data handling and effective communication between agents. Below is a summary of the outputs for each task:

\begin{itemize}
    \item \textbf{Receive and Assess Call Task Output:} Captures critical incident details including fire type, location, injured details, severity, hazards, indoor/outdoor status, and trapped individuals.
    \item \textbf{Notify Other Crews Task Output:} Adds to the \textit{Call Assessment} model to include information about 
    whether medical services are required.
\end{itemize} 