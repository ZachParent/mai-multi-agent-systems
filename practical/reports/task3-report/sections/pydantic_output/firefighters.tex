\subsubsection{Pydantic Outputs}

Structured outputs ensure effective communication and accountability among team members in the Firefighter Agent Crew. Below are the Pydantic models designed to encapsulate outputs for each task in the firefighting process:

\paragraph{Receive Report Task Output}
\begin{lstlisting}[caption={Pydantic model for Receive Report Task Output}]
class FireAssessment(BaseModel):
    location: Location  # Coordinates (x, y)
    fire_type: FireType  # Type of fire fire_severity
    fire_severity: FireSeverity  # Severity of fire: low, medium, or high
    trapped_people: int  # Number of trapped individuals
    hazards: List[HazardType]  # Hazards present
    hazards_present_indoor: bool  # True if fire is indoor, False otherwise
\end{lstlisting}


\paragraph{Allocate Firefighting Resources Task Output}
\begin{lstlisting}[caption={Pydantic model for Allocate Firefighting Resources Task Output}]
class FireFightingMaterial(BaseModel):
    material_name: Literal[
        "pickup_truck",
        "ladder_engine",
        "water_tanker",
        "foam_tanker",
        "dry_chemical_tanker",
        "air_tanker",
    ]
    material_quantity: int

class AllocatedFirefightingResources(BaseModel):
    fire_assessment: FireAssessment
    resources: List[FireFightingMaterial]
\end{lstlisting}


\paragraph{Deploy Fire Combatants Task Output}
\begin{lstlisting}[caption={Pydantic model for Deploy Fire Combatants Task Output}]
class FirefightingActivity(BaseModel):
    firefighting_activity: str
    priority: Literal["low", "medium", "high"]

class DeployedFireCombatants(BaseModel):
    fire_assessment: FireAssessment
    firecombatants_deployed: int
    estimated_arrival_time: datetime
    firefighting_activities: List[FirefightingActivity]
\end{lstlisting}


\paragraph{Report Firefighting Response Task Output}
\begin{lstlisting}[caption={Pydantic model for Report Firefighting Response Task Output}]
class FirefightersResponseReport(BaseModel):
    summary: str
    timestamp: datetime
\end{lstlisting}

\paragraph{Summary of Outputs}
\begin{itemize}
\item \textbf{Receive Fire Report Task Output:} Captures the essential details from the initial fire report, including fire type, severity, hazards, and any trapped individuals.
\item \textbf{Allocate Firefighting Resources Task Output:} Documents the allocation of firefighting materials, including quantities and resource types.
\item \textbf{Deploy Fire Combatants Task Output:} Tracks the deployment of personnel, estimated arrival times, and prioritized firefighting activities.
\item \textbf{Report Firefighting Response Task Output:} Summarizes the firefighting response plan.
\end{itemize} 