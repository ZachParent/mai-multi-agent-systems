\section{Environment}
\label{sec:environment}

This section presents an analysis of the multi-agent system environment in the context of emergency response. 


\subsection{Accessibility}

The environment is \textbf{partially observable}. In a partially observable environment, agents cannot obtain complete, accurate,
and up-to-date information on all aspects of the environment. Here, each agent may have limited knowledge about the current state of other agents,
such as their exact locations or the resources they are carrying at any given time. For example, a fire truck may not know the precise location or
availability of ambulances in the vicinity. This limitation requires agents to operate with incomplete information and make decisions based on their
local observations and periodic updates.

\subsection{Determinism}

The environment is \textbf{non-deterministic}. Although agents follow specific instructions, the outcomes of their actions are uncertain due to unpredictable
environmental changes and external factors. For example, the fire may spread or intensify, and resource availability (like ambulances or fire trucks) may vary.
This requires agents to continuously adapt to changing conditions, as there is no guaranteed single effect from each action.

\subsection{Episodicity}

The environment is \textbf{non-episodic}. Actions taken by agents have interdependent consequences. For instance, dispatching a fire truck to one location affects
its availability for future incidents. This interconnectedness means agents must consider the consequences of their actions over time, as each decision affects
future outcomes in a continuous sequence rather than isolated episodes.

\subsection{Static vs. Dynamic}

The environment is \textbf{dynamic}. Fires can evolve, spread, or be partially contained over time. Additional injuries may occur, and the status of resources
(such as the availability of hospital beds or medical staff) can change unexpectedly. The dynamic nature of this environment necessitates that the system's
agents are capable of frequent plan updates and re-coordination to respond effectively to evolving situations.

\subsection{Discrete vs. Continuous}

The environment is \textbf{continuous}. The city is represented as a graph with coordinates, which means agents have a wide range of possible positions,
paths, and interactions. Additionally, factors such as fire severity and the capacity of resources are continuously changing, requiring agents to
process ongoing updates and adjust their actions accordingly.


In summary, the emergency response environment for this multi-agent system is partially observable, non-deterministic, non-episodic, dynamic, and
continuous. The agents operating within this system must be designed for adaptability and responsiveness, capable of handling incomplete information,
and prepared to respond to a sequence of interconnected events across a continuously changing city landscape.
