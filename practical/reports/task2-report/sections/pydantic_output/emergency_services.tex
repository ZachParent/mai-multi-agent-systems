\section{Emergency Services Crew}

Structured outputs ensure accurate information handling and effective communication within the Emergency Services Crew. 
Below are the Pydantic models designed for each task's output.

\subsection{Receive and Assess Call Task Output}

\begin{verbatim}

class CallAssessmentOutput(BaseModel):
    fire_type: str  # 'indoor' or 'outdoor'
    location: Tuple[float, float]  # (x, y) coordinates
    trapped_people: int  # M (number of trapped individuals)
    hazards_present: bool  # True or False
    fire_size: str  # 'small', 'medium', or 'large'
    injuries: Tuple[int, List[str]]  # (N, [risk levels])
    timestamp: str
\end{verbatim}

\subsection{Assign Fire Priority Level Task Output}

\begin{verbatim}

class FirePriorityOutput(BaseModel):
    priority_level: str  # 'low', 'medium', or 'high'
    timestamp: str
\end{verbatim}

\subsection{Notify Other Crews Task Output}

\begin{verbatim}

class CrewNotificationOutput(BaseModel):
    fire_fighters_info: dict  # {'location': (x, y), 'priority': str, 'trapped': int}
    medical_services_info: dict  # {'location': (x, y), 'injured_count': int, 'injury_risks': List[str]}
    notification_status: bool
    timestamp: str
\end{verbatim}

This format ensures the clarity and consistency required for seamless integration within the system.
