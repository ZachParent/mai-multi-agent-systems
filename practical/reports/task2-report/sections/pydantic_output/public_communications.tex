\subsection{Public Communication Crew}

Structured outputs are crucial for ensuring clarity, consistency, and seamless integration across tasks. Below are the Pydantic models designed for the tasks in the Public Communication Crew process:

\subsubsection{Receive Report Task Output}
\begin{lstlisting}[caption={Pydantic model for Receive Report Task Output}]
class EmergencyReport(BaseModel):
    call_assessment: CallAssessment  # From Emergency Services Crew
    firefighters_response_report: FirefightersResponseReport  # From Firefighters Crew
    medical_response_report: MedicalResponseReport  # From Medical Services Crew
    timestamp: datetime
    fire_severity: FireSeverity
    location_x: float
    location_y: float
\end{lstlisting}

\subsubsection{Search Related Cases Task Output}
\begin{lstlisting}[caption={Pydantic model for Search Related Cases Task Output}]
class RelatedCase(BaseModel):
    case_id: int
    fire_severity: FireSeverity
    location_x: float
    location_y: float
    summary: str

class RelatedCases(BaseModel):
    emergency_report: EmergencyReport
    related_cases: List[RelatedCase]
\end{lstlisting}

\subsubsection{Draft Initial Article Task Output}
\begin{lstlisting}[caption={Pydantic model for Draft Initial Article Task Output}]
class DraftArticle(BaseModel):
    emergency_report: EmergencyReport
    title: str
    public_communication_report: str
\end{lstlisting}

\subsubsection{Integrate Additional Information Task Output}
\begin{lstlisting}[caption={Pydantic model for Integrate Additional Information Task Output}]
class IntegratedArticle(BaseModel):
    emergency_report: EmergencyReport
    public_communication_report: str
    integrated_sources: List[str]
\end{lstlisting}

\subsubsection{Review and Authorize Publication Task Output}
\begin{lstlisting}[caption={Pydantic model for Review and Authorize Publication Task Output}]
class ReviewedArticle(BaseModel):
    emergency_report: EmergencyReport
    public_communication_report: str
    mayor_approved: bool
    mayor_comments: str
\end{lstlisting}

\subsubsection{Provide Social Media Feedback Task Output}
\begin{lstlisting}[caption={Pydantic model for Provide Social Media Feedback Task Output}]
class PublicCommunicationReport(BaseModel):
    emergency_report: EmergencyReport
    public_communication_report: str
    mayor_approved: bool
    mayor_comments: str
    social_media_feedback: str
    timestamp: datetime
\end{lstlisting}

\paragraph{Summary of Outputs}
\begin{itemize}
    \item \textbf{Receive Report Task Output:} Captures the initial fire incident report relevant details from \textit{Emergency Services Crew}, \textit{Firefighters Crew}, and \textit{Medical Services Crew}.
    \item \textbf{Search Related Cases Task Output:} Retrieves relevant historical cases for contextualization and save this case.
    \item \textbf{Draft Initial Article Task Output:} Records the initial draft content.
    \item \textbf{Integrate Additional Information Task Output:} Updates the draft with integrated sources and revisions.
    \item \textbf{Review and Authorize Publication Task Output:} Specifies the review status and comments from the Mayor.
    \item \textbf{Provide Social Media Feedback Task Output:} Details feedback posted on social media platforms, he can critize the mayor's decission.
\end{itemize}
