\subsection{Emergency Services Crew}

The Emergency Services Crew follows a structured \textbf{sequential process} to ensure prompt handling of emergency calls and
 effective communication with response units. The crew consists of two agents: the \textbf{Emergency Call Agent} and the 
 \textbf{Notification Agent}. Each task is assigned based on the specific role and capabilities of these agents, as indicated in the 
 subsequent description.

\begin{enumerate}
    \item \textbf{Receive and Assess Call.} The \textit{Emergency Call Agent} receives incoming calls and collects relevant details
     about the incident. The information that this agent receives is the answer of the following six questions, and it saves the information
     in a report.
     \begin{itemize}
        \item Is it an indoor or outdoor fire? The answer will be outdoor or indoor.
        \item Where is it? The location is received as coordinates (x,y).
        \item Is anyone inside or trapped? The answer will be an integer number $M$ with the number of trapped people. If $M>0$, then rescues are needed and 
        the \textit{Notification Agent} will detail that to the Fire Fighters Crew.
        \item Are there hazards? The answer will be a boolean: True (yes) or False (no). Examples of hazards could be gas cylinders, 
        chemicals, explosions, etc.
        \item How big is the fire? The fire will be classified as either large (e.g. spreading rapidly), medium (e.g. smoke visible) or 
        small (e.g. small flame), depending on the assess of the \textit{Emergency Call Agent.}
        \item Is anyone injured? How badly? The answer will be a tuple with an integer number $N$ representing the number of injured 
        people and another tuple with the level of risk of each person. They could be classified based on their injuries as 'high risk',
         'risk', or 'out of risk'. If there is no victims then $N=0$ and the second element of the tuple will be an empty list.
    \end{itemize}
    
    \item \textbf{Assign fire priority level.} The \textit{Emergency Call Agent} categorizes the incident and sends initial notifications 
    to the \textit{Notification Agent}. The emergency can be classified in one of the following levels: 
    \begin{itemize}
        \item \textbf{Classifying the Fire Based on the Answers:}  
        The fire is classified into three levels based on the answers:
        \begin{itemize}
            \item \textbf{High Priority:}
            \begin{itemize}
                \item Indoor fire or outdoor fire with significant hazards (e.g., chemicals, gas).
                \item $M > 0$ (people are trapped inside).
                \item Large fire spreading rapidly.
                \item High risk injuries or multiple victims.
            \end{itemize}
            The reason of that is because a fire that is indoors, spreading quickly, with trapped people and significant hazards, 
            is clearly high priority. This combination requires an immediate and large response to prevent further harm or fatalities.
            
        \item \textbf{Medium Priority:}
            \begin{itemize}
                \item Outdoor fire with no hazards or small to medium size fire.
                \item No trapped people, but some people at risk or with moderate injuries.
                \item Medium risk injuries or a small number of victims with manageable injuries.
            \end{itemize}
            It was consider that fires that are moderately dangerous, but not life-threatening, or where injuries are serious but
             not critical, would fall into medium priority. These need quick response, but they don't have the same urgency as a high 
             priority fire.
            
        \item \textbf{Low Priority:}
            \begin{itemize}
                \item Outdoor fire with no hazards and small size.
                \item No trapped people or minimal injuries.
                \item Low risk injuries (minor burns or cuts).
            \end{itemize}
            This fire is less dangerous, with no immediate risk to life or property. A single fire engine may be enough to contain 
            and manage the situation. The priority level is low because the fire does not pose significant harm to anyone involved.
    \end{itemize}

    \item \textbf{Notify other crews.} The \textit{Notification Agent} receives the report and the classification of the fire, and 
    details the information to the appropriate crews, Medical Services Crew and Fire Fighters Crew.
    \begin{itemize}
        \item The following information is provided to the Fire Fighters:
        \begin{itemize}
            \item Location (x,y)
            \item Priority level (low, medium or high).
            \item Number of trapped people $M$.
        \end{itemize}
        \item Regarding the Medical Services, they receive the following information:
        \begin{itemize}
            \item Location (x,y)
            \item Number of injured people $N$.
            \item A list containing the risk of each injured person (out of risk, risk or high risk).
        \end{itemize}
    \end{itemize}
\end{enumerate}

\textbf{Task Dependencies} The Emergency Services Crew operates with interdependent tasks that follow a strict 
\textbf{sequential process}:
\begin{itemize}
    \item \textbf{Assign fire priority level} depends on the completion of \textbf{Receive and Assess Call}.
    \item \textbf{Notify other crews.} depends on the completion of \textbf{Categorize and Notify Response Units}.
\end{itemize}

%\begin{figure}[h]
%    \centering
%    \includegraphics[width=\textwidth]{emergency_services_flow.pdf}
%    \caption{Sequential Process Flow of the Emergency Services Crew with Agent Responsibilities}
%\end{figure}
