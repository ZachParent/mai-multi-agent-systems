\begin{enumerate}

    \item \textbf{Receive and Assess Call.} 
    The \textit{Emergency Call Agent} receives incoming calls and collects relevant details
    about the incident. The information that this agent receives answers the following six questions and is saved
    in a report:
    \begin{itemize}
        \item Is it an indoor or outdoor fire? The answer will be either \textit{outdoor} or \textit{indoor}.
        \item Where is it? The location is received as coordinates \((x, y)\).
        \item Is anyone inside or trapped? The answer will be an integer number $M$ representing the number of trapped people. 
              If $M > 0$, rescues are needed, and the \textit{Notification Agent} will detail that to the Fire Fighters Crew.
        \item Are there hazards? The answer will be a boolean: \textit{True} (yes) or \textit{False} (no). Examples of hazards could include 
              gas cylinders, chemicals, explosions, etc.
        \item How big is the fire? The fire will be classified as either \textit{large} (e.g., spreading rapidly), \textit{medium} (e.g., smoke visible), 
              or \textit{small} (e.g., small flame), based on the assessment by the \textit{Emergency Call Agent}.
        \item Is anyone injured? How badly? The answer will be a tuple containing an integer $N$ representing the number of injured 
              people and another tuple listing the risk level of each person. They could be classified based on their injuries as \textit{high risk}, 
              \textit{risk}, or \textit{out of risk}. If there are no victims, then $N = 0$ and the second element of the tuple will be an empty list.
    \end{itemize}
    
    \item \textbf{Assign Fire Priority Level.} 
    The \textit{Emergency Call Agent} categorizes the incident and sends initial notifications 
    to the \textit{Notification Agent}. The emergency can be classified into the following levels:
    \begin{itemize}
        \item \textbf{Classifying the Fire Based on the Answers:}  
        The fire is classified into three levels:
        \begin{itemize}
            \item \textbf{High Priority:}
            \begin{itemize}
                \item Indoor fire or outdoor fire with significant hazards (e.g., chemicals, gas).
                \item $M > 0$ (people are trapped inside).
                \item Large fire spreading rapidly.
                \item High-risk injuries or multiple victims.
            \end{itemize}
            Fires meeting these criteria require immediate and large-scale responses to prevent further harm or fatalities.
            
            \item \textbf{Medium Priority:}
            \begin{itemize}
                \item Outdoor fire with no hazards or small to medium size fire.
                \item No trapped people, but some individuals at risk or with moderate injuries.
                \item Medium-risk injuries or a small number of victims with manageable injuries.
            \end{itemize}
            Fires that are moderately dangerous, but not life-threatening, or where injuries are serious but not critical, 
            fall into this category. These require a quick response but lack the urgency of high-priority fires.
            
            \item \textbf{Low Priority:}
            \begin{itemize}
                \item Outdoor fire with no hazards and small size.
                \item No trapped people or minimal injuries.
                \item Low-risk injuries (minor burns or cuts).
            \end{itemize}
            These fires are less dangerous, with no immediate risk to life or property. A single fire engine may be sufficient to manage the situation.
        \end{itemize}
    \end{itemize}

    \item \textbf{Notify Other Crews.} 
    The \textit{Notification Agent} receives the report and fire classification, then communicates the information to the appropriate crews 
    (Medical Services Crew and Fire Fighters Crew):
    \begin{itemize}
        \item Information provided to the Fire Fighters:
        \begin{itemize}
            \item Location \((x, y)\)
            \item Priority level (\textit{low}, \textit{medium}, or \textit{high})
            \item Number of trapped people $M$
        \end{itemize}
        \item Information provided to the Medical Services Crew:
        \begin{itemize}
            \item Location \((x, y)\)
            \item Number of injured people $N$
            \item A list of the risk levels of each injured person (\textit{out of risk}, \textit{risk}, or \textit{high risk})
        \end{itemize}
    \end{itemize}

\end{enumerate}
