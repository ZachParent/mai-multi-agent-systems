\documentclass[12pt,a4paper]{article}
\usepackage[utf8]{inputenc}
\usepackage[top=1in, bottom=1in, left=1in, right=0.75in]{geometry}
\usepackage{graphicx}
\usepackage{amsmath}
\usepackage{enumerate} % Import the enumerate package
\usepackage{amsfonts}
\usepackage{hyperref}
\usepackage{float}
\usepackage{listings}
\usepackage{xcolor}
\usepackage{verbatim}


% Listings settings for Python code
\lstset{
    language=Python,
    basicstyle=\ttfamily\small,
    keywordstyle=\color{blue}\bfseries,
    stringstyle=\color{red},
    commentstyle=\color{green!70!black},
    numberstyle=\tiny\color{gray},
    numbers=left,
    stepnumber=1,
    numbersep=8pt,
    showspaces=false,
    showstringspaces=false,
    frame=single,
    breaklines=true,
    breakatwhitespace=true,
    tabsize=4,
    captionpos=b
}

\title{Emergency Response: Cooperation and Coordination Mechanisms in Multi-Agent Systems}
\author{Sheena Maria Lang, Antonio Lobo Santos, Zachary Parent, \\ María del Carmen Ramírez Trujillo and Bruno Sánchez Gómez}
\date{\today}

\begin{document}

\maketitle
\tableofcontents
\newpage

\section{Introduction}
In this report, we present the proposed cooperation and coordination mechanisms for the CrewAI emergency response problem. The mechanisms are structured into three main components:
\begin{enumerate}
    \item Process Definition for individual crews.
    \item Pydantic Outputs for structured data handling.
    \item Agent Interaction between different crews.
\end{enumerate}



\section{Process Definition}

\subsection{Public Communication Crew Sequential Process}

The Public Communication Crew operates within a structured sequential process to ensure efficient and accurate communication of fire incident reports to the public. Each task is assigned to a specific agent with well-defined responsibilities, as detailed below:

\begin{enumerate}
	\item \textbf{Receive Report:} The \textit{Communication Operator} obtains the fire incident report in Markdown format. This serves as the starting point for the process and can filter any information that is not relevant for this crew.
	\item \textbf{Search Related Cases:} The \textit{Archive Keeper} searches for past incidents with similar locations or fire types. This task depends on the completion of the \textit{Receive Report} task.
	\item \textbf{Draft Initial Article:} The \textit{Article Writer} drafts an initial article based on the current report. This task also depends on the completion of the \textit{Receive Report} task.
	\item \textbf{Integrate Additional Information:} The \textit{Article Writer} integrates insights from related cases into the draft. This task requires the completion of both the \textit{Search Related Cases} and \textit{Draft Initial Article} tasks.
	\item \textbf{Review and Authorize Publication:} The \textit{Mayor} reviews the article and either authorizes publication or provides feedback for revisions. This task depends on the completion of the \textit{Integrate Additional Information} task.
	\item \textbf{Provide Social Media Feedback:} The \textit{Social Media Commentator} critiques the emergency response in a humorous yet constructive manner. This task depends on the approval of the article by the \textit{Mayor}.
\end{enumerate}

\begin{figure}[h!]
	\centering
	\includegraphics[width=0.9\textwidth]{figures/PC-process.png}
	\caption{Sequential Process Flow of the Public Communication Crew with Agent Responsibilities}
	\label{fig:public_comm_flow}
\end{figure}


\paragraph{Task Dependencies}
The sequential process relies on strict task dependencies to ensure an organized workflow:
\begin{itemize}
	\item \textit{Search Related Cases} and \textit{Draft Initial Article} can be executed in parallel but both depend on \textit{Receive Report}.
	\item \textit{Integrate Additional Information} requires the completion of both \textit{Search Related Cases} and \textit{Draft Initial Article}.
	\item \textit{Review and Authorize Publication} depends on \textit{Integrate Additional Information}.
	\item \textit{Provide Social Media Feedback} requires article approval from the \textit{Mayor}.
\end{itemize}

The visual representation in Figure~\ref{fig:public_comm_flow} highlights these dependencies and assigns colors to denote the responsible agents, ensuring clarity and accountability.




\section{Pydantic Outputs}
Structured outputs are essential for ensuring clarity and consistency in task execution. Below are listed the Pydantic models used in the system.

\subsection{Emergency Services Crew}
Structured outputs ensure accurate information handling and effective communication within the Emergency Services Crew. Below 
are the Pydantic models designed for each task's output.

\subsubsection{Receive and Assess Call Task Output}
\begin{lstlisting}[caption={Pydantic model for Receive and Assess Call Task Output}] 
class EmergencyDetails(BaseModel):
    fire_type: FireType  # Type of fire (e.g., ordinary, electrical, gas, etc.)
    location: Location  # Coordinates (x, y)
    injured_details: List[InjuryType]  # List of risk levels of injured people
    fire_severity: FireSeverity  # Severity of fire: low, medium, or high
    hazards: List[HazardType]  # Hazards present, e.g., gas cylinders, chemicals
    indoor: bool  # True if fire is indoor, False otherwise
    trapped_people: int  # Number of people trapped (0 if none)
\end{lstlisting}

\subsubsection{Notify Other Crews Task Output}
\begin{lstlisting}[caption={Pydantic model for Notify Other Crews Task Output}] 
class CallAssessment(BaseModel):
    fire_type: FireType
    location: Location
    injured_details: List[InjuryType]
    fire_severity: FireSeverity
    hazards: List[HazardType]
    indoor: bool
    trapped_people: int
    medical_services_required: bool  # True if medical services are required, False otherwise
\end{lstlisting}

\paragraph{Summary of Outputs}
The Pydantic outputs for the \textit{Emergency Services Crew} ensure structured data handling and effective communication between agents. Below is a summary of the outputs for each task:

\begin{itemize}
    \item \textbf{Receive and Assess Call Task Output:} Captures critical incident details including fire type, location, injured details, severity, hazards, indoor/outdoor status, and trapped individuals.
    \item \textbf{Notify Other Crews Task Output:} Adds to the \textit{Call Assessment} model to include information about 
    whether medical services are required.
\end{itemize}

\subsection{Firefighter Agent Crew}

Structured outputs ensure effective communication and accountability among team members in the Firefighter Agent Crew. Below are the Pydantic models designed to encapsulate outputs for each task in the firefighting process:

\subsubsection{Receive Report Task Output}
\begin{lstlisting}[caption={Pydantic model for Receive Report Task Output}]
class FireAssessment(BaseModel):
    location: Location  # Coordinates (x, y)
    fire_type: FireType  # Type of fire fire_severity
    fire_severity: FireSeverity  # Severity of fire: low, medium, or high
    trapped_people: int  # Number of trapped individuals
    hazards: List[HazardType]  # Hazards present
    hazards_present_indoor: bool  # True if fire is indoor, False otherwise
\end{lstlisting}


\subsubsection{Allocate Firefighting Resources Task Output}
\begin{lstlisting}[caption={Pydantic model for Allocate Firefighting Resources Task Output}]
class FireFightingMaterial(BaseModel):
    material_name: Literal[
        "pickup_truck",
        "ladder_engine",
        "water_tanker",
        "foam_tanker",
        "dry_chemical_tanker",
        "air_tanker",
    ]
    material_quantity: int

class AllocatedFirefightingResources(BaseModel):
    fire_assessment: FireAssessment
    resources: List[FireFightingMaterial]
\end{lstlisting}


\subsubsection{Deploy Fire Combatants Task Output}
\begin{lstlisting}[caption={Pydantic model for Deploy Fire Combatants Task Output}]
class FirefightingActivity(BaseModel):
    firefighting_activity: str
    priority: Literal["low", "medium", "high"]

class DeployedFireCombatants(BaseModel):
    fire_assessment: FireAssessment
    firecombatants_deployed: int
    estimated_arrival_time: datetime
    firefighting_activities: List[FirefightingActivity]
\end{lstlisting}


\subsubsection{Report Firefighting Response Task Output}
\begin{lstlisting}[caption={Pydantic model for Report Firefighting Response Task Output}]
class FirefightersResponseReport(BaseModel):
    summary: str
    timestamp: datetime
\end{lstlisting}

\paragraph{Summary of Outputs}
\begin{itemize}
\item \textbf{Receive Fire Report Task Output:} Captures the essential details from the initial fire report, including fire type, severity, hazards, and any trapped individuals.
\item \textbf{Allocate Firefighting Resources Task Output:} Documents the allocation of firefighting materials, including quantities and resource types.
\item \textbf{Deploy Fire Combatants Task Output:} Tracks the deployment of personnel, estimated arrival times, and prioritized firefighting activities.
\item \textbf{Report Firefighting Response Task Output:} Summarizes the firefighting response plan.
\end{itemize}

\subsection{Medical Services Crew}

Structured outputs ensure consistency and facilitate effective collaboration among agents within the Medical Services Crew. Below are the Pydantic models for each task's output:

\subsubsection{Receive Report Task Output}
\begin{lstlisting}[caption={Pydantic model for Receive Report Task Output}]
class MedicalAssessment(BaseModel):
    location: Location  # Coordinates (x, y)
    injured_details: List[InjuryType]  # List of risk levels of injured people
    fire_severity: FireSeverity  # Severity of fire: low, medium, or high
    hazards: List[HazardType]  # Hazards present, e.g., gas cylinders, chemicals
\end{lstlisting}

\subsubsection{Rank Hospitals Task Output}
\begin{lstlisting}[caption={Pydantic model for Rank Hospitals Task Output}]
class Hospital(BaseModel):
    hospital_id: str
    location: Location
    available_beds: int
    available_ambulances: int
    available_paramedics: int

class RankedHospitals(BaseModel):
    medical_assessment: MedicalAssessment
    ranked_hospitals: List[Hospital]
    timestamp: datetime
\end{lstlisting}

\subsubsection{Allocate Hospital Resources Task Output}
\begin{lstlisting}[caption={Pydantic model for Allocate Hospital Resources Task Output}]
class HospitalResources(BaseModel):
    hospital_id: str
    beds_reserved: int
    ambulances_dispatched: int
    paramedics_deployed: int

class AllocatedHospitalResources(BaseModel):
    medical_assessment: MedicalAssessment
    hospital_resource_allocation: List[HospitalResources]
    timestamp: datetime
\end{lstlisting}

\subsubsection{Deploy Paramedics Task Output}
\begin{lstlisting}[caption={Pydantic model for Deploy Paramedics Task Output}]
class MedicalEquipment(BaseModel):
    equipment_name: Literal[
        "oxygen_mask",
        "stretcher",
        "defibrillator",
        "IV_drip",
        "other",
    ]
    use_case: str


class DeployedParamedics(BaseModel):
    medical_assessment: MedicalAssessment
    total_paramedics_deployed: int
    total_ambulances_dispatched: int
    estimated_arrival_times: List[datetime]
    equipment: List[MedicalEquipment]
\end{lstlisting}

\subsubsection{Report Medical Response Task Output}
\begin{lstlisting}[caption={Pydantic model for Report Medical Response Task Output}]
class MedicalResponseReport(BaseModel):
    summary: str
    timestamp: datetime
\end{lstlisting}

\paragraph{Summary of Outputs}
\begin{itemize}
    \item \textbf{Receive Report Task Output:} Captures the key details of the fire incident, including injury data.
    \item \textbf{Rank Hospitals Task Output:} Ranks the available hospital based on distance to the emergency site.
    \item \textbf{Allocate Hospital Resources Task Output:} Summarizes the resources provided by each hospital for emergency medical care.
    \item \textbf{Deploy Paramedics Task Output:} Reports the deployment plan, estimated times of arrival of each ambulance, and special medical equipment to be brought.
    \item \textbf{Report Medical Response Task Output:} Provides an overall response plan.
\end{itemize}

\subsection{Public Communication Crew}

Structured outputs are crucial for ensuring clarity, consistency, and seamless integration across tasks. Below are the Pydantic models designed for the tasks in the Public Communication Crew process:

\subsubsection{Receive Report Task Output}
\begin{lstlisting}[caption={Pydantic model for Receive Report Task Output}]
class EmergencyReport(BaseModel):
    call_assessment: CallAssessment
    firefighters_response_report: FirefightersResponseReport
    medical_response_report: MedicalResponseReport
    timestamp: datetime
    fire_severity: FireSeverity
    location_x: float
    location_y: float
\end{lstlisting}

\subsubsection{Search Related Cases Task Output}
\begin{lstlisting}[caption={Pydantic model for Search Related Cases Task Output}]
class RelatedCase(BaseModel):
    case_id: int
    fire_severity: FireSeverity
    location_x: float
    location_y: float
    summary: str


class RelatedCases(BaseModel):
    related_cases: List[RelatedCase]
\end{lstlisting}

\subsubsection{Draft Initial Article Task Output}
\begin{lstlisting}[caption={Pydantic model for Draft Initial Article Task Output}]
class DraftArticle(BaseModel):
    title: str
    public_communication_report: str
\end{lstlisting}

\subsubsection{Integrate Additional Information Task Output}
\begin{lstlisting}[caption={Pydantic model for Integrate Additional Information Task Output}]
class IntegratedArticle(BaseModel):
    public_communication_report: str
    integrated_sources: List[str]
\end{lstlisting}

\subsubsection{Review and Authorize Publication Task Output}
\begin{lstlisting}[caption={Pydantic model for Review and Authorize Publication Task Output}]
class ReviewedArticle(BaseModel):
    public_communication_report: str
    mayor_approved: bool
    mayor_comments: str
\end{lstlisting}

\subsubsection{Provide Social Media Feedback Task Output}
\begin{lstlisting}[caption={Pydantic model for Provide Social Media Feedback Task Output}]
class PublicCommunicationReport(BaseModel):
    public_communication_report: str
    mayor_approved: bool
    mayor_comments: str
    social_media_feedback: str
\end{lstlisting}

\paragraph{Summary of Outputs}
\begin{itemize}
    \item \textbf{Receive Report Task Output:} Captures the initial fire incident report relevant details from \textit{Emergency Services Crew}, \textit{Firefighters Crew}, and \textit{Medical Services Crew}.
    \item \textbf{Search Related Cases Task Output:} Retrieves relevant historical cases for contextualization and save this case.
    \item \textbf{Draft Initial Article Task Output:} Records the initial draft content.
    \item \textbf{Integrate Additional Information Task Output:} Updates the draft with integrated sources and revisions.
    \item \textbf{Review and Authorize Publication Task Output:} Specifies the review status and comments from the Mayor.
    \item \textbf{Provide Social Media Feedback Task Output:} Details feedback posted on social media platforms, he can critize the mayor's decission.
\end{itemize}


\section{Crew Interactions and Flow}
\label{sec:crew_interaction}

\subsection{Design and Coordination}
The Emergency Planner Flow is designed to handle emergency situations by coordinating multiple crews. The flow begins with the retrieval of a call transcript, followed by the processing of the call by emergency services. Based on the assessment, firefighters and medical services are dispatched in parallel. Public communication is managed after both teams report or during approval retries. Once the emergency is resolved, the flow concludes with the generation of a comprehensive emergency report, which includes summaries and timestamps from all participating crews.

\subsubsection{State Management}
The system maintains a centralized state using a Pydantic model \footnote{\url{https://docs.crewai.com/concepts/flows\#structured-state-management}}, `EmergencyPlannerState`, which tracks all aspects of the emergency response. This includes the call transcript, assessments, and response reports. The state model ensures type-safe storage and accommodates partial updates.

\subsection{Implementation}
The flow is orchestrated using CrewAI's decorators, which define the sequence and conditions for crew operations. Key flow control points include:

\begin{itemize}
    \item \texttt{@start()}\footnote{\url{https://docs.crewai.com/concepts/flows\#start}} for initiating the call transcript retrieval.
    \item \texttt{@listen()}\footnote{\url{https://docs.crewai.com/concepts/flows\#listen}} for establishing dependencies between operations, such as emergency services processing and the dispatch of firefighters and medical services.
    \item \texttt{@router()}\footnote{\url{https://docs.crewai.com/concepts/flows\#router}} for handling conditional flow control, particularly for public communication approval.
\end{itemize}


\subsubsection{Router}
The router manages public communication approval, checking if the mayor has approved the communication. If not, it retries up to a maximum count. This ensures that public communication is handled appropriately and efficiently.
This includes the use of \texttt{and\_} and \texttt{or\_} to combine multiple conditions. This is key for the retry mechanism for public communication approval.

\paragraph{Complex Logic for Public Communications}
\texttt{and\_} \footnote{\url{https://docs.crewai.com/concepts/flows\#conditional-logic-and}} and \texttt{or\_} \footnote{\url{https://docs.crewai.com/concepts/flows\#conditional-logic-or}} are used to combine multiple conditions. This is key for the retry mechanism for public communication approval.

\begin{lstlisting}[language=Python]
@listen(or_(and_(firefighters, medical_services), "retry_public_communication"))
def public_communication(self):
    # ...
\end{lstlisting}

\paragraph{Router Logic for Public Communication Approval}
The router emits different messages based on the conditions, either triggering a retry or saving the full emergency report.

\begin{lstlisting}[language=Python]
@router(public_communication)
def check_approval(self):
    logger.info("Checking approval")
    if self.state.public_communication_report.mayor_approved:
        return "save full emergency report"
    elif self.state.mayor_approval_retry_count >= MAX_MAYOR_APPROVAL_RETRY_COUNT:
        return "save full emergency report"
    self.state.mayor_approval_retry_count += 1
    return "retry_public_communication"
\end{lstlisting}

\subsection{Justification of Design Choices}
The design choices are justified by the need for a robust and flexible system that can handle complex emergency scenarios. The use of CrewAI's flow decorators allows for clear and maintainable code, while the parallel processing capabilities ensure timely responses from different crews.



\section{Conclusion}
This report outlines the proposed cooperation and coordination mechanisms for the CrewAI MAS. Future work includes extending the interaction model to include additional agent types and testing the scalability of the system.


\end{document}
