\section{Agents and Tasks}
\begin{frame}{Agents and Tasks}
    \begin{itemize}
        \item For each crew, we define agents and tasks using \texttt{.yaml} files.
        \item Each crew is represented as a Python class, with agents and tasks defined as methods within the class.
        \item Agents are instantiated with configuration settings, and tasks are created by linking them with specific tools and data models.
    \end{itemize}
    \begin{block}{Example Code: MedicalServicesCrew}
        \begin{minipage}[t]{0.48\textwidth} % 48% width for the first listing
            \centering
            \begin{lstlisting}[language=Python, breaklines=true]
medical_services_operator:
  role: Medical Services Operator
  goal: >
    Facilitate communication and coordination within the Medical Services crew and with other crews,
    ensuring efficient information flow and timely updates during emergencies.
  backstory: >
    Trained to handle emergency communication and coordination, the Medical Services Operator excels
    at synthesizing critical information and relaying it accurately. With a background in crisis management,
    this agent ensures that all team members are well-informed and can act swiftly.
  allow_delegation: false
  verbose: true
  llm: ollama/llama3.1
  temperature: 0.7
  max_tokens: 1200
            \end{lstlisting}
        \end{minipage}
        \hfill % Space between the two minipages
        \begin{minipage}[t]{0.48\textwidth} % 48% width for the second listing
            \centering
            \begin{lstlisting}[language=Python, breaklines=true]
rank_hospitals:
  description: >
    1. Receive input containing the Medical Assessment:
            \end{lstlisting}
        \end{minipage}
    \end{block}
\end{frame}
